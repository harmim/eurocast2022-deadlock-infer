%-----------------------------------------------------------------------------
% Eurocast 2022 on Deadlocks -- extended abstract
%-----------------------------------------------------------------------------

\documentclass[runningheads]{llncs}

% Packages
\usepackage{pbox}
%\usepackage{times}
\usepackage{crimson}
\usepackage{color}

% Page style
% \pagestyle{plain}
\pagestyle{plain}

% Section counter (number subsubsections)
\setcounter{tocdepth}{3}

% Useful commands (todo, remark, ...)
\newcommand{\todo}[1]{\textcolor{blue}{[TODO: #1]}}
\newcommand{\rem}[1]{\textcolor{blue}{[NOTE: #1]}}
\newcommand{\fixme}[1]{\textcolor{red}{[FIXME: #1]}}

%-----------------------------------------------------------------------------
\begin{document}
%-----------------------------------------------------------------------------

\mainmatter

\newcommand{\thetitle}{Static Deadlock Detection in Low-Level C Code}

\title{\thetitle\vspace*{-4mm}}

\titlerunning{\thetitle}

\author{
  Dominik Harmim \and
  Vladim\'{\i}r Marcin \and
  Lucie Svobodov\'{a} \and
  Tom\'{a}\v{s} Vojnar\vspace*{-2mm}
}

\institute{
  FIT, Brno University of Technology, Czech Republic
  %% \\
  %% \email{\{iharmim,vojnar\}@fit.vut.cz}
}

\maketitle

%-----------------------------------------------------------------------------
\vspace*{-4mm}\section{Introduction}\vspace*{-2mm}
%-----------------------------------------------------------------------------

Nowadays, programs often use \emph{multithreading} to better utilise the many
processors of current computers.
%
However, concurrency does not only bring speed-ups but also a much larger space
for nasty errors easy to cause but difficult to find.
%
The reason why finding errors in concurrent programs is particularly hard is
that their concurrently running threads may \emph{interleave} in many different
ways, with bugs hiding in just a few of them.
%
Such interleavings need not be discovered even by many-times repeated testing.


Coverage of such rare behaviours can be improved by approaches such as
\emph{systematic testing} \cite{schedspec12} and \emph{noise-based testing}
\cite{noise14}.
%
% \cite{contestframework03,anaconda}.
%
Another way is the use of \emph{extrapolating dynamic checkers}, such as
\cite{fasttrack09}, 
%
% \cite{atomsetser08,velodrome08,realdeadlock09,fasttrack09}, 
%
which can report warnings about possible errors even if those are not seen in
the testing runs, based on spotting some of their symptoms.
%
Unfortunately, even though such checkers have proven quite useful in practice,
they can of course still miss errors.
%
Moreover, monitoring a run of large software through such checkers may be quite
expensive too.

On the other hand, approaches based on \emph{model checking}, i.e., exhaustive
state space exploration, can guarantee discovery of all potentially present
errors---either in general or at least up to some bound, which is usually given
in the number of context switches.
%
The scalability of these techniques is, however, so far not sufficient to handle
truly large industrial code even when combining it with methods such as
\emph{sequentialisation} \cite{lazy-seq-16}, 
%
% \cite{lal-reps-08,abou-seq-11,lazy-seq-16}, 
%
which represents one of the most scalable approaches in the area.

An alternative to the above approaches, which can scale better than model
checking and can find bugs not found dynamically (though for the price of
potentially missing some errors and/or producing false alarms) is offered by
approaches based on \emph{static analysis}, e.g., in the form of \emph{abstract
interpretation} or \emph{data-flow analysis}.
%
The former approach is supported, e.g., in \emph{Facebook Infer}---an
open-source framework for creating highly scalable, compositional, incremental,
and interprocedural static analysers based on abstract interpretation
\cite{inferNFM15}.
%
% Facebook Infer has grown considerably, but it is still under active
% development.
%
% It is employed every day not only in Facebook itself but also in other
% companies, such as Spotify, Uber, Mozilla, or Amazon. 
%
Currently, Infer provides several analysers that check for various types of
bugs, such as buffer overflows, null-dereferencing, or memory leaks.
%
% However, most importantly, Facebook Infer is a~\emph{framework} for building
% new analysers quickly and easily.
%
As for \emph{concurrency-related bugs}, Infer provides a support for finding
some forms of \emph{data races} and \emph{deadlocks}, but it is limited to
high-level Java and C++ programs \cite{racerD18,inferCACM19}.
%
In this work, we propose a \emph{deadlock checker} that fits the common
principles of analyses used in Infer and is applicable to \emph{C code} with
\emph{lower-level lock manipulation}.
%
Our checker is called L2D2 for ``low-level deadlock detection''.

%!!!!!!!!!!!!!!!!!!!!!!!!!!!!!!!!!!!!!!!
\enlargethispage{6mm}
%!!!!!!!!!!!!!!!!!!!!!!!!!!!!!!!!!!!!!!!

Out of existing static deadlock checkers, L2D2 is probably closest to RacerX
\cite{racerX03}, and some of the approaches used in L2D2 are inspired by RacerX.
%
However, unlike L2D2, RacerX uses a context-sensitive analysis, meaning that
that each function needs to be reanalysed in a new context, which reduces the
scalability.

%-----------------------------------------------------------------------------
\vspace*{-3mm}\section{Static Deadlock Detection in Low-Level Concurrent C
Code}\vspace*{-2mm}
%-----------------------------------------------------------------------------

For scalability reasons, L2D2 runs \emph{along the call tree} of a given
program, \emph{starting from its leaves}.
%
Each function is analysed just once without any knowledge of its possible call
contexts.
%
For each analysed function, we derive a \emph{summary} that consists of a
\emph{pre-condition} and \emph{post-condition}.
%
The summaries are then used when analysing functions higher up in the call
hierarchy.
%
Intuitively, the pre-condition states what states of locks the function expects
from its callers, and the post-condition reflects the effect of the function on
the locks.

More precisely, the pre-condition of a function states which locks are
\emph{expected to be locked or unlocked} upon a call of the function to avoid
possible double locking or unlocking, respectively.
%
The post-condition contains information about which locks \emph{may be locked or
unlocked} at the exit of the function.
%
Next, the summary contains so-called \emph{lock dependencies} in the form of
pair of locks $(l_1,l_2)$ where locking of $l_2$ was seen while $l_1$ was
locked.
%
At the end of the analysis, detection of cycles in the lock dependencies is used
to detect possible deadlocks.
%
Finally, the summary also contains information on which \emph{locks may be
locked and then again unlocked} within the given function, which is needed for
detecting lock dependencies with such locks in functions higher up in the call
hierarchy.

L2D2 further implements a number of heuristics intended to decrease the number
of possible false alarms.
%
For example, since double-locking/unlocking errors are quite rare, their
detection may instead be used as an indicator that the analysis is
over-approximating too much, and it may reset (some of) the working sets.
%
An example of another heuristic used is detection of so-called \emph{gate
locks}, i.e., locks guarding other locks (upon which deadlocks on the nested
locks are not reported).

Within our experimental evaluation of L2D2, we have applied it on a set of 1,002
C programs with POSIX threads derived from a Debian GNU/Linux distribution,
originally prepared for evaluating the static deadlock analyser based on
the~\textsc{Cprover} framework proposed in~\cite{kroening16}.
%
The~benchmark consists of 11.4\,MLOC.
%
Eight of the programs contain a~known deadlock.
%
Like \textsc{Cprover}, \textsc{L2D2} was able to detect all the deadlocks.
%
On the remaining 994 programs, it produced 39 false alarms (78 of the programs
failed to compile since some of the constructions used are not supported by the
Infer front end).
%
We find this very encouraging taking into account that the Cprover's deadlock
detector produced 114 false alarms, 453 timeouts (wrt our 30 minutes time
limit), and ran out of the available 24 GB of memory in 135 cases.
%
Altogether, L2D2 consumed less than 1\,\% of the time taken by Cprover.


%-----------------------------------------------------------------------------
% Acknowledgement
%-----------------------------------------------------------------------------

%!!!!!!!!!!!!!!!!!!!!!!!!!!!!!!!!!!!!!!!
\enlargethispage{5mm}
%!!!!!!!!!!!!!!!!!!!!!!!!!!!!!!!!!!!!!!!

\vspace*{2mm}{\fontsize{8}{8}\selectfont \noindent\emph{Acknowledgement.} The
work was supported by the project No. 20-07487S of the Czech Science
Foundation.}\vspace*{-2mm}

%-----------------------------------------------------------------------------
% Bibliography
%-----------------------------------------------------------------------------

%\bibliographystyle{abbrv}
%\bibliography{bibliography}

\begin{thebibliography}{1}
\vspace*{-2mm}

\bibitem{racerD18}
S.~Blackshear, N.~Gorogiannis, P.~O'Hearn, and I.~Sergey.
\newblock {RacerD: compositional static race detection}.
\newblock {\em Proc. of the ACM on Programming Languages},
  2(OOPSLA):144:1--144:28, 2018.

\bibitem{inferNFM15}
C.~Calcagno, D.~Distefano, J.~Dubreil \emph{et al}. 
% D.~Gabi, P.~Hooimeijer, M.~Luca, P.~O'Hearn, I.~Papakonstantinou, J.~Purbrick, 
% and D.~Rodriguez.
\newblock {Moving Fast with Software Verification}.
\newblock In {\em Proc. of NFM'15}, LNCS 9058. Springer, 2015.

\bibitem{inferCACM19}
D.~Distefano, M.~F\"{a}hndrich, F.~Logozzo, and P.~O'Hearn.
\newblock {Scaling Static Analyses at Facebook}.
\newblock {\em Commun. ACM}, 62(8):62--70, 2019.

\bibitem{racerX03}
D.~Engler and K.~Ashcraft.
\newblock {RacerX: Effective, Static Detection of Race Conditions and
  Deadlocks}.
\newblock In {\em Proc. of SOSP'03}. ACM, 2003.

\bibitem{noise14}
J.~Fiedor, V.~Hrub\'{a}, B.~K\v{r}ena, Z.~Letko, S.~Ur, and T.~Vojnar.
\newblock {Advances in Noise-based Testing}.
\newblock {\em Software Testing, Verification and Reliability}, 24(7):1--38,
  2014.

\bibitem{fasttrack09}
C.~Flanagan and S.~Freund.
\newblock {FastTrack: Efficient and Precise Dynamic Race Detection}.
\newblock In {\em Proc. of PLDI'09}. ACM, 2009.

\bibitem{kroening16}
D.~Kroening, D.~Poetzl, P.~Schrammel, and B.~Wachter.
\newblock {Sound Static Deadlock Analysis for C/Pthreads}.
\newblock In {\em Proc. of ASE'16}. ACM, 2016.

\bibitem{lazy-seq-16}
T.~Nguyen, B.~Fischer, S.~L. Torre, and G.~Parlato.
\newblock {Lazy Sequentialization for the Safety Verification of Unbounded
  Concurrent Programs}.
\newblock In {\em Proc. of ATVA'16}, LNCS 9938. Springer, 2016.

\bibitem{schedspec12}
J.~Wu, Y.~Tang, H.~Hu, H.~Cui, and J.~Yang.
\newblock {Sound and Precise Analysis of Parallel Programs through Schedule
  Specialization}.
\newblock In {\em Proc. of PLDI'12}. ACM, 2012.

\end{thebibliography}


%-----------------------------------------------------------------------------
\end{document}
%-----------------------------------------------------------------------------
